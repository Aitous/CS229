\item \points{10} {\bf Linear invariance of optimization algorithms}

Consider using an iterative optimization algorithm (such as Newton's method,
or gradient descent) to minimize some continuously differentiable
function $f(x)$.  Suppose we initialize the algorithm at $x^{(0)} = \vec{0}$.
When the algorithm
runs, it will produce a value of $x \in \Re^\di$ for each iteration:
$x^{(1)}, x^{(2)}, \ldots$.

Now, let some non-singular square matrix $A \in \Re^{\di\times \di}$ be given, and
define a new function $g(z) = f(Az)$.  Consider using
the same iterative optimization algorithm to optimize $g$ (with initialization
$z^{(0)} = \vec{0}$). If the values $z^{(1)}, z^{(2)}, \ldots$ produced by this
method necessarily satisfy $z^{(i)} = A^{-1}x^{(i)}$ for all $i$, we say this
optimization algorithm is {\bf invariant to linear reparameterizations}.


\begin{enumerate}
        \item \subquestionpoints{7}  Show that Newton's method (applied to find the
minimum of a function)
is invariant to linear reparameterizations. Note that since $z^{(0)} = \vec{0}
= A^{-1}x^{(0)}$,
it is sufficient to show that if Newton's method applied to $f(x)$
updates  $x^{(i)}$ to  $x^{(i+1)}$, then Newton's method applied to $g(z)$
will update $z^{(i)}=A^{-1}x^{(i)}$ to
$z^{(i+1)}=A^{-1}x^{(i+1)}$.\footnote{Note that for this problem, you must
explicitly prove any matrix calculus identities that you wish to use that are
not given in the lecture notes.} 



        \ifnum\solutions=1
                \begin{answer}
\end{answer}

        \fi

        \item \subquestionpoints{3}  Is gradient descent invariant to linear
reparameterizations?
Justify your answer.


        \ifnum\solutions=1
                \begin{answer}
\end{answer}

        \fi

\end{enumerate}
